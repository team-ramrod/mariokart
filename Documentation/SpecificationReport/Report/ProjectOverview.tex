\chapter{Project Overview}

The goal of this project is to convert an existing electrically powered go-kart into an autonomous vehicle capable of driving itself. This Project has the ultimate goal of circumnavigating a building on the campus without any human interaction.

Autonomous vehicles and more specifically driverless cars is a field of research that has seen a large amount of development in the last few years. This research comes as the increasing power of computers makes the processing of sensors and the decision making required by these vehicles able to happen in real time even when the vehicle is traveling at high speeds.

There are many advantages of this automation. These include improved safety by preventing road accidents caused by human error, the transportation of loads in dangerous areas such as disaster zones and improving traffic flow. For these advantages to be realized however robust programs must be developed that will allow the vehicle to adapt to operate in a varied and continually changing environment with many unknown variables to compensate for.

Recent projects that have happened in this area can be divided into two main areas. Vehicles for surfaced roads and free ranging vehicles. On surfaced roads the major current project is the Google driverless car. This system combines information gathered by Google street view with information from its on board sensor array to drive on highways with no human interaction. To date these vehicles have driven over 200,000 km without any incidents. A second major driverless vehicle project was the VisLab Intercontinental Autonomous Challenge which ran in 2010. In this challenge 4 autonomous vehicles were driven on a 15,000 km trip from Parma, Italy to Shanghai, China with almost no human interact.

 In free ranging vehicles the DARPA grand challenge run by the US military offered contestants a \$2 million prize for anyone whose autonomous vehicle could complete their off road coarse in the shortest time. The winner of this races vehicle in 2007 completed the off road coarse in 4 hours 10 minutes averaging 23 kmph over rough terrain.

All recent large projects into autonomous vehicles have made extensive use of LIDAR (light detection and ranging) sensor that rapidly scans a laser over the environment measuring distances to create a map of the surroundings. These maps can then be used to allow the program to make decisions about its operation based on a very accurate picture of the world. The only disadvantage with the LIDAR is its large price tag when compared to any other form of sensors. A low cost LIDAR that can operate in direct sunlight costs in excess of \$5000. 


