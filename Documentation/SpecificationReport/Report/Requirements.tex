\chapter{Requirements}
\section{Goals}
Given the continuous nature of this project the only constraint upon our progress is time. The issue here is that we may set goals beyond what we have time to achieve, and consequently fail those goals.

To avoid doing this we have split the project's measurable goals into two categories, those we need to achieve to at least deliver a platform for future groups (i.e. implementing drive-by-wire); and those we need to run the go-kart at a behavioral level. As the goals in the second category require the completion of those in the first, they have a lower priority for us. We expect to achieve drive-by-wire control within the deadline, however completing the behavioral control aspect may not be a realistic goal due to the time constraints.

\subsection{We Expect to Achieve}
A fully Drive-by-Wire go-kart controllable over a USB link. Safety constraints, an interface for sensors and low level control of braking, steering and motor power will all be provided by an embedded system. If this is all completed without any progress on the behavioral software then the USB command interface will need to be documented for future groups.

\subsection{We Hope to Achieve}
The abstraction of the control of the karts systems (steering wheel angle, brake force etc) into higher level commands such as speed and direction. The selection and integration of an array of sensors onto the go-kart allowing sufficient information for autonomous control.

An example application proving the go-kart's `autonomous-ness'. This will require a software application running on a laptop, and the sensors connected to the laptop through a micro-controllers. We had oringinally hoped that this application would demonstrate object detection and avoidance however due to time constraints this application will most likely be of a more basic nature that shows off the potential of the system rather then being a solution that demonstrates what autonomous systems are capable of.

\section{Specifics}
Each set of goals mentioned above covers one of the two components we have split the project into. They are the drive-by-wire (DbW) system and the PC-based behavioral control. The specific requirements of each subsystem are defined below.

\subsection{Drive-By-Wire}
In order to achieve full computer control all of the go-kart's mechanical control systems must be replaced with electromechanical actuators interfaced through a micro-controller network. Additionally, the same network must control the go-kart's electric motor and gather feedback from all systems for full, closed-loop control.

Since the go-kart has an electric motor there are two components requiring actuation, the brake and the steering mechanism.

\subsubsection{Braking}
The braking system requires linear actuation to drive the hydraulic  brake. 

\subsubsection{Steering}
The existing steering system uses a rack and pinion system to steer. Converting this into a DbW system requires a low speed, high torque motor driving the pinion. The absolute position of of the motor needs to be tracked so the motor can be run in a closed loop.

\subsubsection{Control}
The control system requires a way to set the position of the steering and brake systems using the actuators outlined above. Additional to this is the control of the of the accelerator. This system requires the control system to interface directly to the microcontroller that is running the electric motor.

\subsection{Sensing and Behavioral Control}
An array of sensors must be attached to the kart that provide all the information the go-kart requires to interact with its enivroment and perform actions without hitting obstacles or requiring human interaction. These sensors must be connected to the main control system. 

To meet what we hope to acheive for this project we must then be able develop a program on a laptop and attach this to the go-kart via a simple to use interface. The program must operate the steering, brakes and accelerator to provide autonomous functions that can be carried out in an uncontrolled enviroment without human interaction. 

