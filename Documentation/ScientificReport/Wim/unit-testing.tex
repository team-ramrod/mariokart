\subsection{Unit Testing}

  As mentioned earlier Unit Testing is currently seeing a major increase in use
  in forward-thinking software development companies, mainly because of
  evangelical Agile development proponents (especially from the Ruby on Rails
  community).  Unfortunately despite this increase for traditional software
  development, the uptake in embedded development projects has been a lot
  slower.

  The basic premise of unit testing is that if you verify that all parts of your
  system work as intended, then the system as a whole will work as intended.  To
  do the verification you write a lot of small \emph{unit} tests to verify
  minimal sections of your code.  By ensuring that all code you write is tested
  by multiple unit tests you can be confident that the code performs as you
  expect.

  Of course this isn't the same thing as being right, unit tests only verify
  that the code does what the test says it should.  To ensure the code does what
  it should do you need to validate your tests.  This is most commonly done
  informally, the test developers know the intended outcome and write the tests
  with the codes final purpose in mind.  For more critical systems an external
  validation can be performed using a method such as modelling.  Our attempts at
  modelling one of the critical sections of our system can be read about in
  \emph{Safety by Design for the Mariokart System} by Simon Richards
  \cite{Richards_2011}.

  The big problem with unit testing an embedded system is the very low level of
  abstraction.  One of the big reasons it is so popular in the Ruby on Rails
  community is because of how easy it is to write tests for Ruby thanks to its
  very high level of abstraction.  For embedded development you are likely going
  to want to run a few different testing layers; one testing the very low level
  libraries to ensure the registers are being accessed correctly, one testing
  mid level libraries such as character or LCD displays to make sure they're
  calling the low level libraries properly and one testing the actual
  application code.
