\title{Software Engineering Practices\\ in the Mariokart System}

\author{
  \IEEEauthorblockN{
    Wim Looman\\
    \href{mailto:wgl18@uclive.ac.nz}{\texttt{wgl18@uclive.ac.nz}}
  }

  \vspace{0.5\baselineskip}

  \IEEEauthorblockN{
    \emph{Coauthors}: Simon Richards, Zachary Taylor and Henry Jenkins\\
    \href{mailto:scr52@uclive.ac.nz}{\texttt{scr52@uclive.ac.nz}}
    \href{mailto:zjt14@uclive.ac.nz}{\texttt{zjt14@uclive.ac.nz}}
    \href{mailto:hvj10@uclive.ac.nz}{\texttt{hvj10@uclive.ac.nz}}
  }

  \vspace{0.5\baselineskip}

  \IEEEauthorblockN{
    \emph{Supervisor}: Dr. Andrew Bainbridge-Smith\\
    \href{mailto:andrew.bainbridge-smith@canterbury.ac.nz}{\texttt{andrew.bainbridge-smith@canterbury.ac.nz}}
  }

  \vspace{0.5\baselineskip}

  \IEEEauthorblockA{
    Department of Electrical and Computer Engineering\\
    University of Canterbury\\
    Christchurch, New Zealand
  }
}

\maketitle

\begin{abstract}
  This report details a variety of software engineering practices followed today
  in the world of traditional software development.  It then explores why these
  should be adopted by all engineers, and how they can help embedded development
  specifically.  This is all explored in the context of an autonomous go-kart
  system developed at the University of Canterbury.
\end{abstract}
