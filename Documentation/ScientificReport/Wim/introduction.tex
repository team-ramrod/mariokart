\section{Introduction}

  \subsection{Software Engineering}

    Since this report is aimed at an engineering audience most of you will
    believe that a description of Software Engineering is not really required.
    Unfortunately true Software Engineering is relatively unknown, especially in
    programming courses run in Electrical departments around the world.  That is
    not to say that Computer Science departments do a better job of teaching it,
    in fact Software Engineering really should be taught as a subset of
    Engineering \cite{Parnas_1999}, just that the style of programming taught to
    Electrical students is generally light on following the engineering
    practices that the rest of their courses rely on.

    So, what is Software Engineering?  It is simply the application of standard
    Engineering practice to the development of software.  However because of the
    nature of software as a much more fluid abstract thing than the normal
    circuits designed by Electrical Engineers the precise method of application
    has to be changed.
    
    At the same time as being more abstract than a circuit software is also much
    more concrete; there are no (or at least very few) annoying real world
    effects directly on the software.  Assuming the circuit a microprocessor is
    in has been designed well the Software Engineer can take it for granted that
    the digital I/O used by something like a Inter-Integrated Circuit
    (I\textsuperscript{2}C) is basically a perfect connection straight to the
    internals of another device.  Internally if there are no weird defects in
    the microcontroller you can assume that a function like:

\begin{lstlisting}
int return_three() {
    int three = 3;
    return three;
}
\end{lstlisting}

    will always return exactly 3.  Not 2 when the batteries start running low,
    not 4 when it is a particularly hot day, always exactly 3.

    This exactness of software enables the use of a few techniques that are not
    normally available in most engineering professions.  For example it is
    possible to perform exhaustive testing and/or modelling of the system within
    acceptable time.

  \subsection{Mariokart}

    The system on which this report will base most of the examples was codenamed
    Mariokart.  This was a final year project for the University of Canterbury's
    Bachelor of Engineering degree carried out by the authors.  The overall goal
    of the project was to take one of the electric go-karts the department had
    and retrofit a drive-by-wire system on to it with an overall goal of having
    the kart autonomously drive around the campus.  For the purposes of this
    report the main details are:

    \renewcommand{\labelitemi}{}
    \begin{itemize}
      \item 
        The overall design is a distributed system, 5 boards are used; one for
        communication with a host laptop, one for steering, one for brakes, one
        to interface to the motor controller and one for collecting data from a
        variety of sensors.

      \item
        Each board is running an Atmel SAM7XC microprocessor.

      \item
        
    \end{itemize}

    For more details see \emph{Embedded Hardware Design For Autonomous Electric
    Vehicle} by Henry Jenkins \cite{jenkins_2011}.
