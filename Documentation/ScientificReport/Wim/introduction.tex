\section{Introduction}

  \subsection{Software Engineering}

    Since this report is aimed at an engineering audience most of you will
    believe that a description of Software Engineering is not really required.
    Unfortunately, true Software Engineering is relatively unknown, especially in
    programming courses run in Electrical departments around the world.  That is
    not to say that Computer Science departments do a better job of teaching it
    --- they don't \cite{Shepard:2001:MTT:376134.376180} and in fact Software
    Engineering really should be taught as a subset of Engineering
    \cite{Parnas_1999} --- just that the style of programming taught to
    Electrical students is generally light on following the engineering
    practices that the rest of their courses rely on.

    So, what is Software Engineering?  It is simply the application of standard
    Engineering practice to the development of software.  However, because of the
    nature of software as a much more fluid abstract thing than the normal
    circuits designed by Electrical Engineers the precise method of application
    has to be changed \cite{vanVliet:2000:SEP:352372}.

    At the same time as being more abstract than a circuit, software is also
    much more concrete; there are no (or at least very few) annoying real world
    effects directly on the software.  Assuming the circuit a microprocessor is
    in has been designed well the Software Engineer can take it for granted that
    the digital I/O used by something like a Inter-Integrated Circuit
    (I\textsuperscript{2}C) is basically a perfect connection straight to the
    internals of another device.  Internally if there are no weird defects in
    the microcontroller you can assume that a function like the one shown in
    Listing \ref{function} will always return exactly 3.  Not 2 when the
    batteries start running low, not 4 when it is a particularly hot day, always
    exactly 3.

    This exactness of software enables the use of a few techniques that are not
    normally available to most engineering professions.  For example it is
    possible to perform exhaustive testing and/or modelling of the system within
    acceptable time.

    The major components of software engineering that will be discussed in this
    report are: version control, unit testing and continuous integration.
    Version control is probably the aspect of software engineering that is best
    applied by current engineers, however most still use an old system such as
    Subversion despite their being much better alternatives like Git and
    Mercurial available
    \cite{Glassy:2006:UVC:1089182.1089195,Gurbani:2005:CSO:1083258.1083264}.
    Unit testing is a developmental practice that has been seeing a major
    increase in use for traditional software development during recent years.
    This is largely because of development processes such as Test-Driven and
    Behaviour-Driven Development evangelised by the Agile Software Development
    proponents \cite{Muller:2001:CSE:381473.381536}.  Continuous integration is
    another major aspect of these newer software development methods involving
    continuous application of quality control to the system while under
    development, normally utilising unit testing as the main quality assurance
    system \cite{Holck}.

    A lot of this is standard practice in software development shops.
    Unfortunately despite the large amount of code written by other engineering
    disciplines the same level of engineering practice they apply in designing
    their circuit board, concrete floor or ethanol extractor doesn't get applied
    to the code they develop in pursuit of these goals.  This is most relevant
    to embedded development where the entire range of software engineering
    practices can be applied; some take a bit more effort because of the lower
    abstraction level, but they are all applicable in some way.  However parts
    of this are also relevant for the other engineering disciplines; Matlab may
    not be a real programming language, but when developing simulations in it
    proper software engineering practices should still be followed.

\begin{lstlisting}[float,caption={Small function example.},label=function]
int return_three() {
    int three = 3;
    return three;
}
\end{lstlisting}

  \subsection{Mariokart}

    The system on which this report will base most of the examples was codenamed
    Mariokart.  This was a final year project for the University of Canterbury's
    Bachelor of Engineering degree carried out by the authors.  The aim of the
    project was to take one of the electric go-karts the department had and
    retrofit a drive-by-wire system on to it, with an overall goal of having the
    kart autonomously drive around the campus.  For the purposes of this report
    the main details of the system developed are:

    \renewcommand{\labelitemi}{$\bullet$}
    \renewcommand{\labelitemii}{$\circ$}
    \begin{itemize}
      \item The overall design is a distributed system with 5 boards:
            \begin{itemize}
              \item One for communication with a host laptop.
              \item One for steering.
              \item One for braking.
              \item One to interface to the motor controller.
              \item One for collecting data from a variety of sensors.
            \end{itemize}

      \item Each board is running an Atmel SAM7XC microprocessor.

      \item Communication between boards is carried over a Controller Area
            Network (CAN) bus.

    \end{itemize}

    For more details of the system development see \emph{Embedded Hardware
    Design For Autonomous Electric Vehicle} by Henry Jenkins \cite{jenkins_2011}
    and for the autonomous goal see \emph{Development of a Marker Tracking
    System for use in the Mariokart System} by Zachary Taylor \cite{Taylor_2011}.
