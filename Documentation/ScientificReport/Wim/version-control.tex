\subsection{Version Control}

  Out of all Software Engineering practices Version Control is definitely the
  most widely used by normal engineers.  Unfortunately it is still not used
  everywhere.  For example, you have just finished looking at a new power
  regulation system that you want to switch to for your next PCB revision; now
  you need to write a quick report on why it is so much better than the
  current system that you should spend all this time changing your design.
  What's the first thing you should do, open a new Microsoft Word document?
  Load up your \LaTeX\ editor?  \emph{No}, you should initialise a new
  repository or ensure you have the projects documentation repository
  available and updated.  Anything and everything that is more than a few
  lines long, or will ever be shared with a team member should be under
  version control.

  This is very important for a multitude of reasons.  Firstly it provides you
  with a time line of development activity.  If you need to revisit a decision
  months later you can identify exactly when the initial review was made and
  when any revisions happened.
  
  Secondly it provides you with a safety net.  The following anecdote is a
  very good example of why this safety net is important.

  \begin{bigquote}

    {A younger programmer asked an elder about his code and his coding style,
    and how the older programmer would do certain things. The older programmer
    said `Let's take a look at your code', so the younger took out his laptop,
    opened his editor, and showed him.}

    \vspace{5pt}

    {The older programmer looked at the code, thought about it for a bit, and
    then started editing it. He deleted the class internals, leaving only the
    structure, and then rearranged the structure, saying `Here's how I would
    do it to make it more efficient and readable'. After he was done, he saved
    the file and gave it back to the younger programmer, who was ashen-faced.}

    \vspace{5pt}

    {`That... My code is gone!' said the younger programmer. `But you have it
    in version control somewhere, right?' asked the elder. `N.... no.' was the
    reply. `Well then,' said the older, `now you've learned two lessons.'}

    \vspace{5pt}
    
    --- \emph{Dan Udey} \cite{Udey_2008}.

  \end{bigquote}

  Having this safety net is also a very good incentive to experiment.  As
  mentioned in the anecdote the older programmer, assuming that the younger
  was using version control, felt free to delete most of the code and
  rearrange what remained.  If the younger had been using version control then
  they could have simply committed this major change on a separate branch and
  checked out their prior work to compare the two.
