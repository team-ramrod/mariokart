\subsection{Continuous Integration}

  Continuous Integration (CI) is another software development practice highly
  pushed in the Agile development community.  The main goal behind CI is to
  minimize the time between an error being introduced in the code base and the
  error being detected and fixed.  This is achieved by having developers
  continuously integrating their changes and verifying the integrated code via
  an automatic test.  To ensure this is happening a few key steps have to be
  taken; testing the code has to be almost painless, it must be easy for the
  developers to merge their work back into the master development branch and if
  a bug is ever introduced the developer responsible is not allowed to ignore
  it.

  Ensuring that testing the code is almost painless can sometimes be a big task.
  Especially if this was not a priority at the beginning of development, build
  times and test suites have a tendency to expand out of control very quickly.
  This is absolutely necessary for CI to work well though.  To keep the error
  detection time as small as possible you really want all developers to be
  running the test suite after every change, so a maximum run time of two to
  three minutes is required.  For large systems this can be achieved by
  splitting the test suite up, you just need to be careful to ensure all tests
  relating to any change will be run as part of that section of test suite.

  Continuously integrating developers code requires having a single
  \emph{master} development branch.  Whenever a developer decides to do some
  work they will grab the latest version of this branch and start their work
  form there.  Because their team is performing Continuous Integration they can
  be confident that the master branch will be in a working state, or if not that
  whoever broke it knows and will have a fix pushed up to it within a few
  minutes.  They will then write a test for the feature they are going to be
  implementing, followed by the code to make the feature work.  This will be
  repeated until they have finished implementing the feature.  If for some
  reason this is a very large feature that is going to take more than a day or
  two they will try and aim for a few checkpoints consisting of just a few hours
  work.  At each of these checkpoints they will merge in any changes that have
  happened to the master branch and ensure that none of them introduce any
  problems with the feature they're currently implementing.  It is this
  integration of the work other developers are doing multiple times through the
  day that gives Continuous Integration its name.  Once the feature is finished
  it is merged back into the master development branch and pushed out for other
  developers to use.

  Once a feature has been merged into master other developers will be pulling it
  down and integrating it into their current work.  If there is a bug in it then
  it is going to affect a lot of people.  For this reason most teams using CI
  use a Continuous Integration server.  This is simply a server set up to
  continuously pull any change to the master branch, build it, test it, and
  notify the developers if it fails.  In this way if someone accidentally
  commits broken code the CI server will detect it and inform them within a few
  minutes.  At this point the developer has to make getting a fix out their
  highest priority, if they can't see some way to fix it straight away they
  should revert their merge and spend time working out what broke it.
