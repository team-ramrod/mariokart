\section{Performance of Solutions}
The color detection algorithm required easily the least processing power of all the computer vision methods looked at running at a solid 20 fps only limited by the speed of the camera it was receiving the images from. It also performed the best under motion blur  detecting the marker even when it was moved extremely quickly in front of the camera. The system however had the major drawback of being far less robust then the other methods of object detection, even small changes in lighting such as adding an additional light to a room caused the hue of the image to change sufficiently to require recalibration. The background also could not contain significant amounts of a similar color or this would be detected instead of the marker. A further problem was that the position calculations epically the depth were extremely crud. Lighting changed the detected size of the object to such an existent that the estimated distance could change by more than 50\% each time the system was setup leading to frequent recalibrations.

The SURF algorithm also performed poorly in many situations. The algorithm ran slowly on the 2.1 GHz test laptop averaging 5 fps. The SURF algorithm was also largely affected by distance and motion blur. On the 340 by 480 pixel test camera the image had to in most cases take up around 50\% of the scene before it was detected, this meant that the distance at which the algorithm would work was limited to less then around 3 meters for it to work even a fraction of the time. The algorithm could also not tolerate motion blur loosing the image as soon as it was moving at even a small rate. However when the algorithm did work it was relatively robust to lighting in comparison to the color detection methods. It also gave locations that under testing were found to be accurate to within 20\%.

Chessboard detection operated slowly if the chessboard had not been located in the previous frame only operating at a rate as low as 2 fps. Once the chessboard was found however the algorithm used the previous location to find it more quickly raising the frame rate to around 15 fps. The chessboard showed some sensitivity to motion blur requiring the corners of the board to be clearly distinguishable for detection to occur. The system was by far the most robust to changes in lighting conditions with detection occurring in all environments without recalibration being required. Its detection of the position was the most accurate of all the tested solutions with an error of less than 10\% being observed in testing.