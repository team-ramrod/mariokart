\section{Performance of Solutions}
. It also performed the best under motion blur  detecting the marker even when it was moved extremely quickly in front of the camera. The background also could not contain significant amounts of a similar color or this would be detected instead of the marker. A further problem was that the position calculations epically the depth were extremely crud. Lighting changed the detected size of the object to such an existent that the estimated distance could change by more than 50\% each time the system was setup leading to frequent recalibrations.

The SURF algorithm also performed poorly in many situations. The algorithm ran slowly on the 2.1 GHz test laptop averaging 3 fps. The SURF algorithm was also largely affected by distance and motion blur. On the 340 by 480 pixel test camera the image had to in most cases take up around 50\% of the scene before it was detected, this meant that the distance at which the algorithm would work was limited to less then around 3 meters for it to work even a fraction of the time. The algorithm could also not tolerate motion blur loosing the image as soon as it was moving at even a small rate. However when the algorithm did work it was relatively robust to lighting in comparison to the color detection methods. It also gave locations that under testing were found to be accurate to within 20\%.

The chessboard showed some sensitivity to motion blur requiring the corners of the board to be clearly distinguishable for detection to occur. The system was by far the most robust to changes in lighting conditions with detection occurring in all environments without recalibration being required. Its detection of the position was the most accurate of all the tested solutions with an error of less than 10\% being observed in testing.

The algorithms were put through a series of tests to guage how well they would perform in the system. A summary of the result of these tests is shown in Table~\ref{results}. All the tests were performed on a test was performed on a 2.1 GHz notebook with an Intel Centrino Duo processor. The webcam was a Creative Labs Laptop Integrated Webcam. This webcam had a resolution of 320 by 240 pixels anh operated at 20 fps.

The robustness to light was tested by setting up an algorithm to recognise its marker in a room with a large amount of natural light. The lighting was then changed by closing the curtains and seeing if the object was still detected, once this was done the rooms lights were turned on. The maximum distance an object could be detected at was. Colour detection performed the worst in this test requiring recalibration before it could detect the target after every change in lighting. This problem steamed from the fact that while the hue of the object being tracked remained constant changing the lighting levels altered the hue the camera perceived the object to be by a small amount. This problem stemmed right from the fundamental method by which the camera detects light and so very little could be done without recalibrating to attempt to account for this change \cite{detect}. The SURF algorithm handled almost all lighting changes without issue. The one situation in which it failed was if enough glare was coming off the object it was tracking to completely obscurer most of the features it was tracking. The chessboard algorithm worked under every tested lighting condition. Large amounts of glare reduced the distance it could be detected from but even in this situation the small chessboard could be detected at up to 4m. An example of this systems robustness is shown in Figure.

The FPS (frames per second) for each system was recorded under normal operation. The color detection algorithm required easily the least processing power of all the computer vision methods looked at running at a solid 20 fps only limited by the speed of the camera it was receiving the images from. The SURF algorithm operated very slowly original operating at 0.8 FPS on locating a 640 by 480 pixel reference image in the scene. This speed was increase to 2 FPS by decreasing the reference image to 320 by 240. This was the lowest quality reference image which could be accurately placed in the scene. Chessboard detection operated slowly if the chessboard had not been located in the previous frame only operating at a rate as low as 2 fps. Once the chessboard was found however the algorithm used the previous location to find it more quickly raising the frame rate to around 15 fps.

Robustness to motion was tested in a fairly subjective manner. The marker was held 1 m from the camera and shaken until detection was lost. With colour detection regardless of how fast the marker was moved detection was never lost, though the detected size decreased by up to around 50\% as the image of the marker became a blur. With SURF detection only slow and careful movements would prevent the camera losing the target. Chessboard detection would lose the board if any quick or jerky movements were made. While this test was quite crud it clearly showed that SURF would be too susceptible to motion blur to be appropriate. It also showed that chessboard detection was also weak to this and a higher quality camera or some pre processing might be needed to help reduce the effect of motion blur when the system is attached to the kart.

Maximum distance the algorithm could detect a marker from was tested next. To make this a fair test for all systems the marker size for each system was set to an A4 page with the colour detections blue ball replaced by a sheet of blue card. The test was performed by walking away from the camera holding the marker and recording the distance at which the target was lost for the first time. The process was repeated three times.  
\begin{center}
    \begin{tabular}{ | l | l | l | p{2.5cm} |}
    \hline
    Algorithm & Colour & SURF & Chessboard \\ \hline
	FPS & 25+ & 3 & 15 \\ \hline
	Robust to light & No & Yes & Yes \\ \hline
	Max distance & 20 m & 2 m & 10 m \\ \hline
	Needs calibration & Yes & No & No \\ \hline
	Robust to motion & Yes & No & handles small movements \\ \hline
    \end{tabular}
\end{center}