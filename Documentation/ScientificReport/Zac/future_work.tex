\section{Future work}
Time constraints on this project meant that not all of the desired functionality was explored and implemented. One area that needed further work was what the path following algorithm did if it had lost sight of the marker and had not relocated it by the time it came to the end of its line of remembered points. This situation could occur if the vehicle it was following was performing a close turn around an obstacle or if it went down into a depression and out again. A possible solution to this would have been to fit a spline to the points the system had been following and extrapolate out from this the most likely coarse the leader with the marker had taken. 

A second possibility that was not followed was the idea of using multiple methods of detection to find the marker. Chessboard detection worked well on its own but its performance suffered during motion blur, had some issues at large distances and had a slow output rate when it lost the target marker. Colour detection had none of these same problems but suffered from far more crude distance measurements and the need to be recalibrated for different light conditions. The strengths of the two systems could have been combined by using the chessboard system to constantly recalibrate the colour detection system using a brightly coloured chessboard instead of a black and white one to act as the marker for both systems. This would mean that the colour detection would have no longer had the issues with lighting and distance location as every time the chessboard was detected it would be used to reset the colour being tracked and to adjust the distance readings so the two systems matched.

Due to time constraints on the project the system was never tested on the kart and because of this no testing on how the system would perform at speed has been done.