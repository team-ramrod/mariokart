\section{Marker Loctaion Methods}

A large variety of location methods were looked at when determining what type of sensors should be perchused for the kart system. Most modern autonomous vechiles rely heavily on scanning laser sensors as their main method of viewing the surrondings. Scanning lasers rapidly move a measuring laser over the enviroment forming a matirix of distances to provide a 3d view of the world. While extremely useful these sensors were discounted early due to their high price tag with even the most basic models starting at \$2000. The Microsoft Kinect camera was briefly looked at as an alternative solution that would provide 3d information about the world. This camera used structured light combined with an ir camera to detect the loctaion of objects in its field of view. This was also quickly discounted however as it had a maximum range of only 3.5m which was too limiting for the application. A second aspect that counted against it was that the ir pattern used by the Kinect was also washed out by direct sunlight making outdoor use impractical.

GPS systems were discounted due to their inability to work reliably in an indoor environment. One solution that was given serious consideration was using a marker that output a RF signal and having three receivers on the kart use its signal strength to triangulate the position of the object. After further reasearch it was found however that distance estimation using the level of RF detectors that were within our budget were only accurate to around 1m under ideal conditions \cite{rf}. This limitation means that the receivers must be placed far apart to minimize the effect the error would have on the direction the triangulation would calculate the signal to have come from