\section{Marker Location Methods}

A large variety of location methods were looked at when determining what type of sensors should be purchased for the kart system. Most modern autonomous vehicles rely heavily on scanning laser sensors as their main method of viewing the surroundings. Scanning lasers as their name implies rapidly move a measuring laser over the environment forming a matrix of distances to provide a 3d view of the world \cite{laser}. While these lasers are extremely useful they were discounted as a possibility early due to their high price tag with even the most basic models starting at \$2000. The Microsoft Kinect camera was briefly looked at as an alternative solution that would provide 3d information about the world \cite{kinect}. This camera uses structured light combined with an IR (infer red) camera to detect the loctaion of objects in its field of view. This system was also quickly discounted however as it had a maximum range of only 3.5m which was too limiting for the application of tracking a marker from on board a kart. A second aspect that counted against the Kinect was that the IR pattern used by it was washed out by direct sunlight making outdoor use impractical.

GPS systems while briefly examined as a possibility were not given any serious consideration due to their inability to work reliably in an indoor environment. One solution that was given serious consideration was using a marker that output a RF (radio frequency) signal and having three RF receivers on the kart use the  relative strength of the signal to triangulate the position of the marker. After further research it was found however that distance estimation using the level of RF detectors that were within the budget were only accurate to around 1m under ideal conditions \cite{rf}. This limitation meant that the receivers would have to be placed far apart to minimize the effect the error would have on the direction the triangulation would calculate the signal to have come from. On a mobile kart system where the close proximity of the sensors was unavoidable this limitation made the method infeasible.

The sensor system that was decided upon to develop was a computer vision system. This system utilized a web-cam mounted to the front of the kart. A unique marker was to be used that could be located in the image. This method had several advantages over the other methods considered. It was a cheap and easy to mount solution with only a web-cam required that would attach straight to the laptop with a USB cable. It provided a large degree of flexibility with a large suite of different algorithms and methods able to be implemented and tested for no modifications to the hardware the system was using. Finally it was the only system that showed promise in being able to locate a marker accurately at distance and within the budget. It was for these reasons that the computer vision system was the only option pursued for development.