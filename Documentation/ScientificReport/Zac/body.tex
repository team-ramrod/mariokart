\section{Body}
%needs splitting up

1.	INTRODUCTION



2.	THEORY OF OPERATION

Computer Vision Tracking Solutions Theory Of Operation



Implementation of Solutions

Color detection, SURF and Chessboard detection were all then implemented to examine their performance for tracking an object in the required environment. To aid in this implementation extensive use was made of the opencv computer vision libraries. These libraries were used as they contained implementations of a large number of computer vision algorithms that had been thoroughly tested by a large community and optimized to run as effectively and efficiently as possible.

The first algorithm implemented was the simple color based approach. In this approach a bright red ball was used as the marker to track. For this approach the scene was thresholded for a value that was approximately red and opening (that is erosion followed by dilation) was used to fill any holes in the image of the ball and remove background noise. A rough estimation of the balls location was then made using its x and y position in pixels from the center of the camera. The z location of the ball was taken to be proportional to the inverse of the size of the ball in pixels. These measures of x, y and z were extremely crude however they provided adequate for the testing that was performed on the color method.

The SURF algorithms marker was a printout of a -------------. This image was used as it contained a large amount of strong lines and detail that meant that a large number of unique and easily identifiable points could be located on the image.

Two chessboards were constructed and tested with the alg

The Path Following Algorithm

Once the kart had located a marker a method for following the marker had to be implemented. The simplest solution of simply driving towards the marker had many drawbacks. An example would be that if the kart was following the marker and the marker moved forwards and then after a few meters turned to the right. The following kart would cut the corner and potentially smash straight into an obstacle. A second problem would be that if the marker left the karts field of vision the kart would be forced to immediately come to a stop as it no longer had anything to track.

The algorithm constantly updated the absolute position of the kart. It did this by using the karts speed sensor and a sensor that gave the angle of the karts wheels. Using this information and recording the time between runs of the kart the position and orientation of the kart could be found. Once the location of the kart had been found the relative position and orientation of the chessboard marker if present was also recorded. After this had been done the absolute position of the marker was calculated. The current and all previous positions were stored in an array.

The core of the path following algorithm was very simple and executed as follows. 
1) The kart would select the first point on the path
2) The kart would set the steering to point at the point and drive towards it
3) When the kart came within a minimum distance of the line (set to the turning radius of the kart)  or the point became too ‘old’ the kart would deem itself to have reached the point
4) The kart would then repeat the process for the next point

By setting the criteria for meeting the point to twice the turning radius of the kart this meant that the kart would always be able to turn to be within this distance  of the next point regardless of where it lay. The kart skipping a point if it became older then a set value (in testing set to 10 seconds) meant that if the kart began to lag behind the object it was tracking it would start cutting corners in order to catch up to the target. This meant that if the target was slowly moving through tight turns the system would work on matching its path and when the target was moving quickly the system would priorities keeping up with the target. 

3.	DISCUSSION

The color detection algorithm required easily the least processing power of all the computer vision methods looked at running at a solid 20 fps only limited by the speed of the camera it was receiving the images from. It also performed the best under motion blur  detecting the marker even when it was moved extremely quickly in front of the camera. The system however had the major drawback of being far less robust then the other methods of object detection, even small changes in lighting such as adding an additional light to a room caused the hue of the image to change sufficiently to require recalibration. The background also could not contain significant amounts of a similar color or this would be detected instead of the marker. A further problem was that the position calculations epically the depth were extremely crud. Lighting changed the detected size of the object to such an existent that the estimated distance could change by more than 50% each time the system was setup leading to frequent recalibrations.

The SURF algorithm also performed poorly in many situations. The algorithm ran slowly on the 2.1 GHz test laptop averaging 5 fps. The SURF algorithm was also largely affected by distance and motion blur. On the 340 by 480 pixel test camera the image had to in most cases take up around 50% of the scene before it was detected, this meant that the distance at which the algorithm would work was limited to less then around 3 meters for it to work even a fraction of the time. The algorithm could also not tolerate motion blur loosing the image as soon as it was moving at even a small rate. However when the algorithm did work it was relatively robust to lighting in comparison to the color detection methods. It also gave locations that under testing were found to be accurate to within 20%.

Chessboard detection operated slowly if the chessboard had not been located in the previous frame only operating at a rate as low as 2 fps. Once the chessboard was found however the algorithm used the previous location to find it more quickly raising the frame rate to around 15 fps. The chessboard showed some sensitivity to motion blur requiring the corners of the board to be clearly distinguishable for detection to occur. The system was by far the most robust to changes in lighting conditions with detection occurring in all environments without recalibration being required. Its detection of the position was the most accurate of all the tested solutions with an error of less than 10 % being observed in testing.

4.	CONCLUSIONS

Three computer vision solutions were looked at for tracking a marker for use in allowing a kart to follow a lead vehicle. Color detection was found to be too sensitive to changes in lighting conditions for it to be used in a robust tracking system. The SURF algorithm ran slowly and its inability to handle even small amounts of motion blur made it inappropriate for the application of being mounted to a moving vehicle. Chessboard detection was the only system that was robust enough to detect a marker and give its position reliability and it was slightly hindered by it own inability to deal with large amounts of motion blur. The output of these systems were combined with sensors measuring the speed and wheel angle of the kart to produce a system that followed the path the marker had taken rather than the marker giving more accurate following. 