\section{Performance of Path Following}

Due to delays in the project converting the go-kart to a point where it could be interfaced with the laptop the algorithm was never tested on the kart. To allow for some basic testing a simple GUI output was instead added to the program. This mapped the abolute location of the kart and the chessboard as well as the path they had taken. It also highlighted the point which the kart was currently driving towards. Fake input from the steering and speed sensors were created giving the kart a constant speed and a maximum turn rate of 30 degrees per second. This interface allowed the system to be debugged and tested though was fairly limited in comparison to the actual go-kart. Its largest problem was that as the system belived that it was moving forward and turning it saw the marker it was tracking infront of it as moving to maintain the same relative position. This meant that testing the systems ability to follow patterns layed out by the marker was hindered.