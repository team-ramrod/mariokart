\section{Introduction}

An electric go-kart was to be converted to allow it to move autonomously. The system required a basic program to control its movement. This program was to function as a proof of concept of the kart’s abilities as well as a method to test the systems that had been added to the kart. It had been decided that this program should allow the kart to follow a marker. The system that was being designed allowed the systems on the kart to interface through a laptop to 5 boards located on different areas of the kart. These boards handled all the low level control of the kart providing the laptop with functions to set the steering, brake and accelerator position. The only sensor present on the kart at the start of the development was a speed sensor however a large number of I/O ports of different types allowed for the addition of extra sensors.  

The developed program was to run on the on-board laptop. It had to operate in real time recognizing the marker, finding its location and outputting the path to travel on with minimal lag. The kart was expected to be able to operate in both indoor and outdoor environments in a large variety of conditions without losing the marker. The system and any additional equipment and sensors required had to be low budget as the total budget for the entire conversion was less then \$1500.