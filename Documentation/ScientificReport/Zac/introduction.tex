\section{Introduction}

An electric go-kart was to be converted to allow it to move autonomously. As a proof of concept of the karts abilities a simple program was to be used in conjunction with sensors placed on the kart to allow the kart to follow a marker placed on a leading vehicle or person. The system that was being designed allowed the systems on the kart to interface through a laptop to 5 boards located on different areas of the kart. These boards handled all the low level control of the kart providing the laptop for functions to set the steering, brake and accelerator position. The only sensor present on the kart at the start of the development was a speed sensor however a large number of I/O ports of different types allowed for the addition of extra sensors.  

The developed program was to run on the on-board laptop. It had to operate in real time recognizing the marker, finding its location and outputting the path to travel on with minimal lag. The kart was expected to be able to operate in both indoor and outdoor environments in a large variety of conditions without losing the marker. The system and any additional equipment and sensors required had to be low budget only costing a total of a few hundred dollars.