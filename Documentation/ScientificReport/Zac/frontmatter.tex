\title{Development of a Marker Location System for use in the Mariokart System}

\author{
  \IEEEauthorblockN{
    Zachary Taylor\\
    \href{mailto:zjt14@uclive.ac.nz}{\texttt{zjt14@uclive.ac.nz}}
  }

  \vspace{0.5\baselineskip}

  \IEEEauthorblockN{
    \emph{Coauthors}: Simon Richards, Wim Looman and Henry Jenkins\\
    \href{mailto:scr52@uclive.ac.nz}{\texttt{scr52@uclive.ac.nz}}
    \href{mailto:wgl18@uclive.ac.nz}{\texttt{wgl18@uclive.ac.nz}}
    \href{mailto:hvj10@uclive.ac.nz}{\texttt{hvj10@uclive.ac.nz}}
  }

  \vspace{0.5\baselineskip}

  \IEEEauthorblockN{
    \emph{Supervisor}: Dr. Andrew Bainbridge-Smith\\
    \href{mailto:andrew.bainbridge-smith@canterbury.ac.nz}{\texttt{andrew.bainbridge-smith@canterbury.ac.nz}}
  }

  \vspace{0.5\baselineskip}

  \IEEEauthorblockA{
    Department of Electrical and Computer Engineering\\
    University of Canterbury\\
    Christchurch, New Zealand
  }
}

\maketitle

\begin{abstract}
  An autonomous go-kart system, nicknamed Mariokart was being developed. It required a simple algorithm to demonstrate its abilities and test its systems. To meet this need an algorithm was developed to follow the path of a marker. Three computer vision techniques color tracking, SURF and chessboard recognition were examined for the identification of the marker. Chessboard detection was chosen mainly due to its robustness to changes in lighting. The system was tested but never mounted to the Mariokart due to delays in the development of systems it required to interface with the kart.
\end{abstract}
