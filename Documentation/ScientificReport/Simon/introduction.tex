
\begin{abstract}
An electric go-kart is outfitted with a drive-by-wire system in the first phase of development of 
an autonomous vehicle. The safety aspects of such are design are evaluated and analysed  
using typical project management techniques. A model of the distributed software required is 
developed and proven and the place of modelling in systems engineering is discussed..
\end{abstract}

\section{Introduction}

\subsection{Project Description}

The authors' reports are based on a group project carried out in the final year 
of their Bachelor of Engineering degree at Canterbury University. The original goal 
of this project was to take one of the electric go-karts used by the department for
power electronics projects and convert it into an autonomous vehicle capable of
a simple navigation task. Due to time restrictions the aim was reduced to implementing
the mechanical and electronic systems needed to control the kart's motion over a USB link.

The authors refer to the project using the moniker Mariokart and this name may be used 
throughout this report in reference to the team, the vehicle or the project itself.

\subsection{System Design}
The overall system design consisted of a laptop, electronics, a CAN bus, actuators and an interface to 
the existing motor driver  hardware.


\begin{itemize}
 \item Electronics: Five PCBs were located around the kart, each in close proximity to 
 the sub-system it was responsible for. The PCBs were identical in layout, specialisation was
 achieved by removing components from boards where they were not needed. For example
 it was intended for only one board to communicate directly with a laptop and so the other
 boards did not have USB headers. The decision to distribute control was made primarily to avoid
 problems caused by driving motors and actuators over a long distance in a high noise environment.
 \item Actuators: The brake pads were driven by a linear actuator mounted in place of the brake pedal. 
 The actuator was mounted to the floor of the go-kart and its driver board was situated nearby. The 
 steering column was similarly controlled by a DC motor mounted in place of the steering wheel 
 and fixed to the frame. The DC motor included an encoder so that the relative angle of the 
 steering column could be determined. Limit switches at each extreme of rotation ensures the
 motor does not turn too far and allows for the absolute angle to be measured.
 \item Controller laptop: To simplify use for future projects, the entire system is controllable via
 commands sent over a USB cable. The communications board is responsible for interfacing with
 a laptop and distributing the user's commands 
 
 \item Motor interface: Power electronics and a motor driver had already been implemented as part 
 of a student project so the only requirement from the motor driver board was to interface with this over
 the available SPI link.

 \item The CAN bus: CAN is a multi-master serial bus with guaranteed sending and message priority. It 
 was designed by Robert Bosch GmBH for networking micro-controllers in automobiles and has
 also some popularity in industrial automation. The CAN specification is low-level only, it does not specify 
 a data protocol or a method address allocation. There are numerous similar software stacks built on top
 of CAN for various purposes such as the Society of Automotive Engineers' J1939\cite{J1939} however due to the simplistic nature of our 
 network we implemented our own protocol.
\end{itemize}

For a more detailed description and justifications see \emph{Embedded Hardware Design For Autonomous Electric
Vehicle} \cite{jenkins_2011}.

\subsection{Risk}
As with any vehicle, autonomous or otherwise, there is always a risk to the user, bystanders and
 the go-kart itself if a part of the system malfunctions in any way. Replacing a human driver with 
 electromechanical control systems removes the non-deterministic element of human error 
 from the system but adds a new set of risks and potential design mistakes.
 
 In this paper, the sources of these risks will be listed analysed and techniques for mitigation and
 negation will be investigated and evaluated.


