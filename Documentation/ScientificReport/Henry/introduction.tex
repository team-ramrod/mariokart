\section{Introduction}
% Goal / problem statement (what is your goal/what problem are you solving)
  Currently the Electrical and Computer engineering department of the
  University of Canterbury has several go-karts. These go-karts were purchased
  as petrol go-karts then retrofitted with control electronics, power
  electronics, lead-acid batteries and a DC electric motor. The goal of this
  project was to take one of these existing electric go-karts and have it drive
  it's self autonomously. 

  An autonomous vehicle is a driver-less vehicle that is able to navigate from
  a location to another set location by controlling actuators and motors. This
  is achieved by gathering data from the surrounding environment in order to
  find and track the vehicle's current location. While the vehicle can have
  communications off the vehicle, like off loading signal processing, it is
  ideal to have the vehicle completely self contained. This, however, does not
  dismiss using incoming radio signals. For example using Global Positioning
  System (GPS) would be perfectly acceptable\cite{yeah,lol}.

  The control systems, both mechanical and electrical, must interface with
  existing hardware on the go-kart. The project aimed to select appropriate
  actuators, motion and distance sensors, develop a navigation system,
  interface to the existing control systems and interface to a central
  computing platform.  As the kart also was required for other projects, it was
  also a requirement to able return the kart to it's initial electric go-kart
  state within an hour.

  To be able to drive a go-kart autonomously, first the kart must be able drive
  according to given digital instructions. So the initial milestone is to build
  embedded hardware systems that are able to control the go-kart and drive
  actuators for both the brake and steering. Because the timeline of this
  project spans approximately 24 weeks, a sub-goal was set of having
  \emph{drive-by-wire} go-kart my the end of the project.

