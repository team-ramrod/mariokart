\title{Embedded Hardware Design For Autonomous Electric Vehicle}

\author{
  \IEEEauthorblockN{
    Henry Jenkins \\
    \href{mailto:hvj10@uclive.ac.nz}{\texttt{hvj10@uclive.ac.nz}}
  }

  \vspace{0.5\baselineskip}

  \IEEEauthorblockN{
    \emph{Coauthors}: Simon Richards, Zachary Taylor and Wim Looman\\
    \href{mailto:scr52@uclive.ac.nz}{\texttt{scr52@uclive.ac.nz}}
    \href{mailto:zjt14@uclive.ac.nz}{\texttt{zjt14@uclive.ac.nz}}
    \href{mailto:wgl18@uclive.ac.nz}{\texttt{wgl18@uclive.ac.nz}}
  }

  \vspace{0.5\baselineskip}

  \IEEEauthorblockN{
    \emph{Supervisor}: Dr. Andrew Bainbridge-Smith\\
    \href{mailto:andrew.bainbridge-smith@canterbury.ac.nz}{\texttt{andrew.bainbridge-smith@canterbury.ac.nz}}
  }

  \vspace{0.5\baselineskip}

  \IEEEauthorblockA{
    Department of Electrical and Computer Engineering\\
    University of Canterbury\\
    Christchurch, New Zealand
  }
}

\maketitle

\begin{abstract}
  This report details the design and production of seven printed circuit boards
  (PCBs) for the purpose of controlling an electric go-kart. The overall goal of
  the project is to build an autonomous go-kart that can drive itself around
  campus. Five of the PCBs communicate via a CAN bus to collect data and control
  other circuitry, while the other two PCBs control actuators for steering and
  brake control. The system is based around Atmel AT91SAM7XC micro controllers
  along with various other electronics. All the PCBs are four layers so were
  manufactured in America by Advanced Circuits, then populated and tested at the
  University of Canterbury.
\end{abstract}
