\section{Conclusions}
  The design process used for the schematic and PCB layout worked well providing
  a set of very functional and reliable PCBs. They have been shown to all
  communicate together via the CAN Bus protocol and drive both the steering and
  brake actuators. 

  Since the project has been in the software development stage for some time
  now, there have only been three mistakes in the hardware found. All of these
  mistakes have had relatively easy fixes. The worst of the three is one side of
  the quadrature clock converter's pins were reversed. This was fixed by
  soldering the IC on upside down and then wiring the pins over to the correct
  pins. The second worst was a miss placed decoupling capacitor on the
  \emph{avref} pin of the SAM7. This was simply replaced with a zero ohm
  resistor to solve the problem. The third problem was that the UART connectors
  don't have a ground pin. To get around this the \emph{sclk} pin in tied low in
  software to act as a ground.

  Using the same design for all of the five main PCBs worked well. Doing this
  saved time in the design, ordering and population. This also saved money by
  being able panellise the boards which would have cost 50 USD more if wasn't
  done. By not having to spend time on making five different designs, more time
  was able to be spent checking for mistakes in the design. If there were
  multiple designs more mistakes would have happened.

  The CAN Bus produced clean signals with all nodes connected. Even with the CAN
  bus running at 1Mb/s its maximum speed. While the bandwidth of the CAN Bus is
  about 1000 times more than required for the current application of a
  drive-by-wire go-kart, it does mean the CAN bus can be expanded for future
  use with sensors and other devices generating traffic on the network. If noise
  were to become a problem, the bus is able to run at a slower rate potentially
  solving this issue.
  \newpage

  The help received from Michael Cusdin made this project possible. His
  knowledge and experience guided us through the PCB population process. Also
  technical knowledge provided by Dr Michael Hayes often lead us out of strife
  with challenging problems.
