\section{Discussion}
 In general the design process produced good stable and usable boards. However
 there were a few areas that needed improving on.

\subsection{Debug Port}
  When the schematic was being set out the debug unit from the SAM7 were left
  with no header leading to them. At the time it was thought that they would be
  unnecessary and then were forgotten about. This meant all the debugging was
  required to be done via a UART port and the display. While this wasn't a bad
  solution to the problem, it would have been better to have access to the debug
  unit.

\subsection{Test points}
  At the last minute in the PCB design about 100 test points were added to the
  bottom of the PCB. This was very useful and one of the better design decisions
  with the PCB. The test points gave access to many different signals on the
  board with test point probes able to be clipped onto them. The only thing that
  should be done differently with the test points is placing more ground test
  points around the board. This would allow more than one oscilloscope cable to
  be hooked up to the ground on the board.

\subsection{Power Supply Selector}
  Making the decision to put battery voltage over the same cables that the CAN
  bus uses made the system work well. However when developing and debugging, it
  would have been ideal to be able to disconnect the power coming in from the
  CAN bus cable. This would mean that an individual board could be power cycled
  without having to power cycle all of them. This could have been done by simply
  putting two jumpers on the PCBs to connect specific ones to enable either the
  CAN bus power or the battery header power.

\subsection{Connectors}
  To save space the Molex Pico Blade \cite{PicoBlade} connectors were used.
  However after using them this was a bad choice. They are very hard to make
  cables for, then brake a short time after. This is frustrating when it takes
  so long to make them, yet they don't last long. While using bigger more
  traditional connectors would have taken up more space on the PCBs, it would
  have saved a lot of time while developing with the boards.

%\subsection{Atmel Library}
  % You suck atmel
